\section{Mathematical background}\label{sec:math}

Vector's rotation is the basis of \cordic{} algorithm. Let's suppose to have an
input vector \((x_0, y_0)\) and that we want to rotate it of an arbitrary angle
\(\theta\). Suppose that we choose the origin as the center of the rotation, so
we can compute the rotated vector as shown in~\eqref{eq:rotatedvector}.

\begin{equation}\label{eq:rotatedvector}
	\begin{cases}
		x_1 = x_0\cos\theta - y_0\sin\theta\\
		y_1 = x_0\sin\theta + y_0\cos\theta
	\end{cases}
\end{equation}

We can take out the factor \(\cos\theta\) and rewrite
the~\eqref{eq:rotatedvector} as shown in~\eqref{eq:rotatedvector2}.

\begin{equation}\label{eq:rotatedvector2}
	\begin{cases}
		x_1 = \cancel{\cos\theta}(x_0 - y_0\tan\theta)\\
		y_1 = \cancel{\cos\theta}(y_0 + x_0\tan\theta)
	\end{cases}
\end{equation}

We cancel out \(\cos\theta\) to approximate the value: instead of making a real
rotation we do a \emph{pseudo-rotation}. Thus, the result of this rotation will
be scaled of a factor \(\frac{1}{\cos\theta} \ge 1\).

There's no a mathematical justification for dropping \(\cos\theta\) but the
reason we do this is that the algorithm is simplified and requires only sums and
subtractions. Moreover, it's possible to get back to the correct non-scaled
value by multiplying the vector by \(\frac{1}{A_n}\), where \(A_n\) is the scale
factor defined as in~\eqref{eq:scalefactor}.

\begin{equation}\label{eq:scalefactor}
	A_n = \prod_{i=0}^{n-1} \frac{1}{\cos\theta_i}
\end{equation}

The scale factor depends only on the number of iterations \(n\) performed by the
algorithm.

If we take angles \(\theta_i\) such that \(\tan\theta_i = 2^{-i}\) we can
rewrite the~\eqref{eq:rotatedvector2} as shown in~\eqref{eq:rotatedvector3}.

\begin{equation}\label{eq:rotatedvector3}
	\begin{cases}
		x_1 = x_0 - y_0 2^{-i}\\
		y_1 = y_0 + x_0 2^{-i}
	\end{cases}
\end{equation}

And the scale factor of the~\eqref{eq:scalefactor} can be rewritten as shown
in~\eqref{eq:scalefactor2}.

\begin{equation}\label{eq:scalefactor2}
	A_n = \prod_{i=0}^{n-1} \sqrt{1 + 2^{-2i}}
\end{equation}

Iteratively, the pseudo-rotation shown in~\eqref{eq:rotatedvector3} can be
written as shown in~\eqref{eq:iterative}.

\begin{equation}\label{eq:iterative}
	\begin{cases}
		x_{i+1} = x_i - d_i(2^{-i}y_i)\\
		y_{i+1} = y_i + d_i(2^{-i}x_i)
	\end{cases}
\end{equation}

Where \(d_i \in \{-1, +1\}\) is a decision operator that is used to choose the
direction of rotation (\(-1\) clockwise, \(+1\) anticlockwise). \(d_i\) is equal
to \(-1\) when \(y < 0\) and equal to \(+1\) otherwise\footnote{What said for
the decision operator is valid only when \cordic{} is used in \emph{Vectoring
Mode}. In this document this is the only mode we will discuss.}.

\cordic{} needs also to keep track of the accumulative angle rotated at each
iteration. The iterative relation shown in~\eqref{eq:angleaccumulator} serves
this purpose.

\begin{equation}\label{eq:angleaccumulator}
	z_{i+1} = z_i - d_i\theta_i
\end{equation}

The final iterative relations of \cordic{} are shown in~\eqref{eq:cordic}.

\begin{equation}\label{eq:cordic}
	\begin{cases}
		x_{i+1} = x_i - d_i(2^{-i}y_i)\\
		y_{i+1} = y_i + d_i(2^{-i}x_i)\\
		z_{i+1} = z_i - d_i\theta_i
	\end{cases}
\end{equation}

The initial values of \(x_0, y_0\) are the Cartesian coordinates of the
point, and \(z_0 = 0\).

The~\eqref{eq:cordic} converges only if the point is in the
1\textsuperscript{st} or 4\textsuperscript{th} quadrant of the Cartesian plane.
For this purpose, when \(x < 0\), \cordic{} needs to pre-rotate the input vector
using the~\eqref{eq:prerotation}.

\begin{equation}\label{eq:prerotation}
	\begin{array}{ll}
		\begin{cases}
			z_0 = 0\\
			x_0 = x_0\\
			y_0 = y_0
		\end{cases} & x \ge 0\\
		\begin{cases}
			z_0 = \pi\\
			x_0 = -x_0\\
			y_0 = -y_0
		\end{cases} & x < 0 \land y \ge 0\\
		\begin{cases}
			z_0 = -\pi\\
			x_0 = -x_0\\
			y_0 = -y_0
		\end{cases} & x < 0 \land y < 0
	\end{array}
\end{equation}

The~\eqref{eq:cordic} converges to the values shown
in~\eqref{eq:cordicconvergence}\footnote{\(n-1\) is the n\textsuperscript{th}
iteration (the last) since we start from \(0\).}.

\begin{equation}\label{eq:cordicconvergence}
	\begin{cases}
		x_{n-1} = A_n \sqrt{x_0^2 + y_0^2}\\
		y_{n-1} = 0\\
		z_{n-1} = \arctan\frac{y_0}{x_0}
	\end{cases}
\end{equation}
