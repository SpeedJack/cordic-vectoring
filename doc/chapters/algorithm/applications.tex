\section{Possible applications}\label{sec:applications}
\cordic{} is used to compute a large number of trigonometric, logarithmic and 
hyperbolic functions, real and complex multiplication, division, square root
calculation, solution of linear systems eigenvalue calculation and many others
mathematical problems.
This versatility and the simplicity is the reason why this algorithm has known a
wide spread in many application areas such as:
\begin{itemize}
	\item Signal and image processing: Discrete Cosine Transform, used in
		compression algorithms, need to compute cosine signals, this can
		be done using \cordic{};
	\item Aerospace and Automotive Application: these application need to 
		track the motion of a vehicle. It can be done using some
		trigonomtric functions.
	\item Communication Systems: CORDIC algorithm can be used for generating
		sine and cosine waves in modulator/demodulators, Phase Locked
		Loop where the use of a Look Up Table for sine and cosine values
		storage can be unfeasible because of the dimension of the ROM,
		thus a simple and powerful arithmetic as \cordic{} can
		efficiently be used to compute these values when needed.
\end{itemize}
