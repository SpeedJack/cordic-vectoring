\section{Possible applications}\label{sec:applications}

\cordic{} is used to compute a large number of trigonometric, logarithmic and
hyperbolic functions, real and complex multiplication, division, square root
calculation, solutions of linear systems, eigenvalue calculation and many others
mathematical problems.

This versatility and the simplicity is the reason why this algorithm has known a
wide spread in many application areas such as:
\begin{description}
	\item [Signal and image processing] Discrete Cosine Transform, used in
		compression algorithms, need to compute cosine signals, this can
		be done using \cordic{};
	\item[Aerospace and automotive applications] in this field, there is the
		need to track the motion of a vehicle. It can be done using some
		trigonometric functions;
	\item[Communication systems] \cordic{} algorithm can be used for
		generating sine and cosine waves in modulator/demodulators,
		Phase Locked Loop where the use of a \emph{look-up table} for
		sine and cosine values storage can be unfeasible because of the
		dimension of the ROM, thus a simple and powerful arithmetic as
		\cordic{} can efficiently be used to compute these values when
		needed.
\end{description}
