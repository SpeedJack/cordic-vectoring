\chapter{Synthesis}\label{ch:synthesis}
After the verification of the correctness of our VHDL model through \matlab{} 
and \modelsim{} we can now proceed with the synthesis and implementation on the
Xilinx's ZyBo Development Board using Xilinx Vivado.
In this section we present the results of our synthesis in terms of resource
utilization, timing and power consumption.

Before to start the analysis of the results we want to clarify some assumption
we have to make in order to correctly understand them:
\begin{itemize}

	\item The results are obtained running first the Synthesis, then the
		Implementation on the board ZYNQ XC7Z010-1CLG400C-1, using the
		default synthesis and strategy provided by Vivado.

	\item The constraints file we've provided specifies only the desired
		clock frequency of 125 MHz, this frequency has been choosen
		without the purpose of a real implementation, just following the
		hints provided during Laboratory lectures.

	\item The pin placement is completely left to Vivado, no pin
		specification has been provided before the Implementation.

	\item The maximum clock frequecy obtained has been computed supposing
		that, after the Implementation, Vivado has found the best
		solution possible.
\end{itemize}	
\section{Timing}
The first thing we are going to analyze is the timing report, produced after the
implementation step, that effectively place the components choosen during the 
synthesis on the board, thus we have a finer estimation, becouse we can take in
account also routing delay.
As we can see from \lstref{timingreport} all the timing constraints are met, and
the Slack both for Clock period and Timing are positive, this means that we can
drive the board to an higher clock frequency then the specified one. More
precisely we can compute the Maximum clock frequency \(f_{max}\) as:
		\[f_{max}= 1/(T_{clk} - WNS) = 222.518914 MHz \]
Where \(T_{clk} = 8.000ns\) that is the period corresponding to 125 MHz.

The \emph{Max Delay Path} is mostly determined by the routing delay as we can 
see from the \lstref{timingreport}, and is in one of the stages of the pipeline 
the \(7^{th}\) to be precise, the logic delay is mostly affected by the carry
logic.

\lstinputlisting[label={lst:timingreport},                                                                                            
 caption={Timing report 
 (\code{timing\_report\_impl.txt}).}]{vivado/timing_report.txt}    

\section{Utilization}
In the utilization report we can appreciate the fact that even if the pipeline
architecture has an higher area consumption then the basic iterative
implementation we use few resource with respect to the one available. Obviously
given by the fact that we do not use any multiplier at the end of the algorithm 
to scale-out the result is makes possible to save some resources,
but several area reduction are also possible thanks to the hardwired shift
operations at each stage and also thanks to the elimination of the ROM to store
all the values of the elementary angles. As we can see in \lstref{utilreport}
for each slice type we do not use more then 4.51\% of the available resources.

The IO resources instead are highly used since we need 34 input pin  and 40 
output pin.
\lstinputlisting[label={lst:utilreport},                                                                                            
 caption={Resource utilization report
 (\code{utilization\_report.txt}).}]{vivado/utilization_report.txt}    

\section{Power Consumption}
Our implementation, as we can see in \lstref{powerreport} as a power consumption
of 172mW, of which 79mW of dynamic power and 92 of static power. As we can see
from the section 1.1 of the report the bigger part of dynamic power is consumed
by the IO ports, where the inner logic as a global dynamic power consumption of
only 32mW.
\lstinputlisting[label={lst:powerreport},                                                                                            
 caption={Power consumption report
 (\code{power\_report.txt}).}]{vivado/power_report.txt}    


\section {Warning Messages}
During the synthesys and implementation process some warning messages have been
shown from Vivado, we will now explain the reason for each one of them
\begin{description}
	\item[Constraints 18-5210: No constraints selected for write]
		This message as said during laboratory lectures can be ignored
	\item[NSTD-1 Critical Warning Unspecified I/O Standard]
		This message arise because we have not specified how the output
		or input must be provided from/to the board. 
	\item[UCIO-1 Critical Warning Unconstrained Logical Port]
		This message arise because we have not specified the pin
		placement, leaving all the job to Vivado.
	\item[ZPS7-1 Warning PS7 block required]
		This message like the previous one arise becouse no pin
		placement specification have been provided to the tool

\end{description}
