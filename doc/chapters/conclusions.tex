\chapter{Conclusions}\label{ch:conclusions}

To conclude we make some consideration on our results.

The synthesis has been done in a fully automatized way, we have left all the
control to the tool, so what can be done to improve the results obtained is to
proceed in an iterative way, giving to \vivado{} everytime higher clock
frequency in order to find the effective maximum clock frequency. Another
consideration is to use other synthesis strategies provided by the tool, for
example we can use a synthesis strategy for high performance, or reduced power
consumption, depending on our constraints.

A good result instead cames out from the VHDL model verification, we can see
that using 20-bits output we can achieve a precision of \(10^{-4}\) on radius
and phase calculation. This give us the measure of the utility of \cordic{}
algorithm, that allows constrained devices to perform computations that
otherwise would be unfeasible, due to power or area demand of a traditional
arithmetic and logic unit.
